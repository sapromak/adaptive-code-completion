\section{Comprehensive Compilation of Evaluation Results}

We present results of our experiments in Table \ref{tab:extended-results}. They can be used as baselines for further research.

We additionally include results for the base model with RoPE's base frequencies ($\theta$) being 10,000 and 500,000, results for pretraining with file-level composer for the same values of $\theta$. These results demonstrate that RoPE adjustments impact model quality, and that the model with initial base frequency performs on zero-level for long contexts even after finetuning.

\begin{table}
\centering

\caption{Results of evaluating all checkpoints after repository-level pretraining on all evaluation composers.
\label{tab:extended-results}
Evaluation dataset is LCA-large for the line categories: \textit{inproject} and \textit{infile}.
Highlighted column is the main column for in-context learning capabilities comparison.}

\resizebox{\textwidth}{!}{
    \begin{tabular}{lc cc>{\columncolor{gray!30}}cc c cc>{\columncolor{gray!30}}cc}
    \toprule
    
    \multirow{3}{*}{\makecell{\textbf{Pretraining} \\ \textbf{Composer}}} & & \multicolumn{4}{c}{\bf inproject} & & \multicolumn{4}{c}{\bf infile} \\\cmidrule(lr){3-6}\cmidrule(lr){8-11}
    & & FL-4K & PD-4K & PD-16K & Or-16K & & FL-4K & PD-4K & PD-16K & Or-16K \\
    
    \midrule
    Base model (no training) & & & & & & & & & & \\
    ~~~$\theta = 10{,}000$ & & 26.4 & 36.6 & 0.0 & --- & & 32.6 & 38.2 & 0.0 & --- \\
    ~~~$\theta = 500{,}000$ & & 13.5 & 16.6 & 9.8 & --- & & 15.5 & 12.9 & 4.5 & --- \\
    
    \cmidrule(lr){1-1}
    File-level $4$K & & & & & & & & & & \\
    ~~~$\theta = 10{,}000$ & & 26.2 & 36.4 & 0.0 & 26.2 & & 32.7 & 38.1 & 0.0 & 32.7 \\
    ~~~$\theta = 500{,}000$ & & 25.9 & 36.1 & 45.2 & 25.9 & & 33.0 & 38.1 & 44.6 & 33.0 \\
    
    \cmidrule(lr){1-1}
    Path Distance \texttt{.py} & & 26.2 & 37.0 & 48.8 & 48.8 & & 33.1 & 38.7 & 47.6 & 48.8 \\
    ~~~\textit{reversed} & & 26.1 & 36.9 & 48.3 & 43.2 & & 32.9 & 38.8 & 47.5 & 44.0 \\
    ~~~\textit{irrelevant} & & 25.8 & 36.5 & 47.9 & 26.7 & & 32.5 & 38.1 & 46.7 & 33.4 \\
    
    \cmidrule(lr){1-1}
    Lines IoU \texttt{.py} & & 25.7 & 36.3 & 48.7 & 51.8 & & 33.2 & 38.4 & 47.7 & 50.1 \\
    ~~~\textit{reversed} & & 26.1 & 36.8 & 48.4 & 43.5 & & 33.2 & 38.9 & 47.4 & 44.6 \\
    ~~~\textit{irrelevant} & & 25.8 & 36.4 & 47.5 & 26.7 & & 32.7 & 38.4 & 46.6 & 33.4 \\
    
    \cmidrule(lr){1-1}
    Code Chunks \texttt{.py} & & 25.9 & 36.5 & 47.9 & 47.8 & & 32.8 & 38.2 & 47.5 & 47.9 \\
    ~~~\textit{reversed} & & 26.1 & 36.5 & 47.8 & 41.3 & & 32.8 & 38.3 & 47.4 & 43.0 \\
    ~~~\textit{irrelevant} & & 25.8 & 36.5 & 47.7 & 26.9 & & 32.3 & 37.9 & 46.3 & 33.2 \\
    
    \cmidrule(lr){1-1}
    Half-memory \texttt{.py} & & 25.7 & 36.0 & 47.4 & 38.6 & & 32.9 & 38.4 & 46.6 & 38.7 \\
    ~~~\textit{reversed} & & 25.7 & 36.2 & 47.3 & 35.0 & & 32.9 & 38.2 & 46.5 & 36.9 \\
    ~~~\textit{irrelevant} & & 25.8 & 36.0 & 47.0 & 27.5 & & 32.4 & 37.7 & 46.5 & 33.0 \\
    
    \cmidrule(lr){1-1}
    Declarations \texttt{.py} & & 25.9 & 36.5 & 46.8 & 28.2 & & 32.6 & 38.1 & 46.1 & 34.4 \\
    ~~~\textit{reversed} & & 25.7 & 36.5 & 46.9 & 28.1 & & 32.7 & 38.1 & 45.7 & 34.2 \\
    ~~~\textit{irrelevant} & & 26.2 & 36.3 & 47.2 & 28.2 & & 32.4 & 38.4 & 45.6 & 33.9 \\
    
    \cmidrule(lr){1-1}
    Text Chunks \texttt{.py} & & 26.1 & 36.6 & 47.5 & 26.9 & & 33.0 & 38.5 & 46.9 & 33.2 \\
    ~~~\textit{reversed} & & 26.0 & 36.1 & 47.4 & 26.8 & & 32.9 & 38.5 & 46.2 & 33.5 \\
    ~~~\textit{irrelevant} & & 25.9 & 36.4 & 47.2 & 26.8 & & 32.7 & 38.5 & 46.2 & 33.8 \\
    
    \cmidrule(lr){1-1}
    Text files & & 25.9 & 36.2 & 47.1 & 26.9 & & 33.0 & 38.6 & 46.4 & 33.5 \\
    ~~~\textit{reversed} & & 26.0 & 36.5 & 46.9 & 26.7 & & 33.2 & 38.4 & 46.2 & 33.1 \\
    ~~~\textit{irrelevant} & & 26.0 & 36.5 & 47.1 & 27.0 & & 32.7 & 38.2 & 46.3 & 33.7 \\
    
    \cmidrule(lr){1-1}
    Random files & & 26.2 & 37.0 & 48.1 & 29.8 & & 32.8 & 38.3 & 47.3 & 34.2 \\
    
    \cmidrule(lr){1-1}
    Random \texttt{.py} & & 25.9 & 36.8 & 48.4 & 31.9 & & 32.8 & 38.1 & 47.0 & 35.3 \\

    \cmidrule(lr){1-1}
    Mixed context & & 26.2 & 36.7 & 48.5 & 31.0 & & 32.6 & 38.2 & 47.5 & 36.4 \\
    
    \midrule
    Random tokens & & 26.0 & 36.2 & 44.5 & 26.0 & & 32.6 & 37.9 & 45.1 & 33.1 \\
    
    \cmidrule(lr){1-1}
    Duplication & & 19.6 & 28.8 & 34.7 & 96.7 & & 24.5 & 27.0 & 28.1 & 95.0 \\
    
    \cmidrule(lr){1-1}
    Leak & & 24.9 & 34.8 & 46.1 & 82.9 & & 30.8 & 35.5 & 43.6 & 81.6 \\
    ~~~\textit{reversed} & & 24.3 & 34.7 & 45.6 & 83.8 & & 30.8 & 35.3 & 42.8 & 81.0 \\
    ~~~\textit{irrelevant} & & 24.5 & 34.6 & 45.6 & 82.2 & & 31.3 & 35.6 & 43.2 & 79.7 \\
    
    \cmidrule(lr){1-1}
    Masked Leak & & 25.2 & 35.4 & 46.4 & 65.5 & & 31.6 & 36.9 & 45.0 & 63.5 \\
    
    \bottomrule
    \end{tabular}
}
\end{table}

% \todo{Write somethin here from commented}



% \subsection{Analysis and Discussion}

When using FL-4K composer, the model successfully recovers its quality after RoPE adjustments, suggesting that file-level data alone is sufficient to restore performance. The initial model shows strong in-context learning capabilities for the PD-4K composer, with it outperforming file-level inference. This advantage persists after repository-level pretraining, indicating that training on the collected data effectively retains model's ability to utilize relevant context for shorter context size.

For the PD-16K composer, the initial model, without RoPE adaptation, fails completely, but RoPE scaling alone improves Exact Match scores. Further pretraining yields gains of +19 for file-level training and +22 for the best composer in \textit{inproject} category, with all final scores being slightly higher than file-level pretraining performance. This suggests that adapting to the longer context window, rather than the specific sequence composition, is the primary factor in repository-level code completion, with context composers contributing only marginally for suggested approaches (+3 points for \textit{inproject} category).

% In the original pretraining composer with 16K truncation setting (column 16K+O), results cluster into two groups: one matching baseline composer performance and another aligning with file-level scores. This indicates that certain composers fail to retrieve meaningful repository-wide context and behave similarly to file-level approaches.

Overall, our findings emphasize that RoPE adaptation is the dominant factor in long-context performance gains, while sequence composition plays a secondary role. Future work should explore more effective retrieval-based strategies to maximize repository-level context utilization.


