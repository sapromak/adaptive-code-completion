% pdflatex
% \usepackage[utf8]{inputenc}
% \usepackage[T1]{fontenc}
% \usepackage[czech]{babel}
% \usepackage{lmodern}

% lualatex
\usepackage{fontspec}
\usepackage{polyglossia}
\usepackage{lmodern}

% math
\usepackage{amsmath}
\usepackage{amsthm}
\usepackage{amssymb}
\usepackage{tikz}
\usetikzlibrary{%
	shapes.geometric,calc,arrows,decorations.markings,decorations.pathmorphing,spy
}
\usepackage{pgfplots}
% \pgfplotsset{compat=1.18}
\usepackage{ifthen}
\usepackage{bbding}								% fancy tick/cross symbol
\usepackage{booktabs}							% fancy tables
\usepackage{datatool}
\usepackage{nicefrac}
\usepackage{fontawesome}
\usepackage{appendixnumberbeamer}
% \usepackage{todonotes}
\usepackage{minted}

\usepackage{graphicx}
\usepackage{listings}
\usepackage{svg}

\lstset{
    basicstyle=\ttfamily\small,
    keywordstyle=\color{blue}\bfseries,
    commentstyle=\color{gray},
    frame=single,
    breaklines=true,
    escapeinside={(*}{*)},
}

% unicode characters
\hypersetup{%
	unicode=true,
	pdfstartview=Fit,
	pdfview=Fit,
	pdfpagemode=UseNone
}

%% Czech-style trigonometric functions
\DeclareMathOperator{\tg}{\mathrm{tg}}
\DeclareMathOperator{\arctg}{\mathrm{arctg}}
\DeclareMathOperator{\arccotg}{\mathrm{arccotg}}
\DeclareMathOperator{\cotg}{\mathrm{cotg}}

%% Other Czech stuff
\newcommand{\uv}[1]{,,#1``}

%% Other custom commands
\DeclareMathOperator{\sgn}{\mathrm{sgn}}
\DeclareMathOperator{\graf}{\mathrm{graf}}
\newcommand{\ceq}{\mathrel{\mathop:}=}
\newcommand{\veps}{\varepsilon}
\newcommand{\dx}{\mathrm{d}x}
\newcommand{\dy}{\mathrm{d}y}
\newcommand{\dz}{\mathrm{d}z}
\newcommand{\dt}{\mathrm{d}t}
\renewcommand{\Re}{\mathrm{Re}\,}
\renewcommand{\Im}{\mathrm{Im}\,}

%% zvyrazneni
\renewcommand{\emph}[1]{\textit{#1}}
\newcommand{\notion}[1]{\textbf{#1}}
\newcommand{\eng}[1]{\textit{#1}}
\newcommand{\tred}[1]{{\color{red}#1}}
\newcommand{\tblue}[1]{{\color{blue}#1}}
\newcommand{\tbrown}[1]{{\color{brown}#1}}
\newcommand{\tgray}[1]{{\color{gray}#1}}
\newcommand{\tgreen}[1]{{\color{green!60!black}#1}}

%% nice link
\newcommand{\nicelink}[2]{\href{#1}{{\small\faicon{external-link}}\,#2}}
\newcommand{\courses}[1]{\href{https://courses.fit.cvut.cz/#1}{{\small\faicon{external-link}}\,#1}}

%% special symbols
\newcommand{\NN}{\mathbb{N}}
\newcommand{\ZZ}{\mathbb{Z}}
\newcommand{\QQ}{\mathbb{Q}}
\newcommand{\RR}{\mathbb{R}}
\newcommand{\eRR}{\overline{\mathbb{R}}}
\newcommand{\pinfty}{+\infty}
\newcommand{\minfty}{-\infty}
\newcommand{\bigO}{\mathcal{O}}
\newcommand{\bigOmg}{\Omega}
\newcommand{\bigomg}{\omega}
\newcommand{\bigth}{\Theta}
\newcommand{\bigo}{o}
\newcommand{\ee}{\mathrm{e}}
\newcommand{\diag}{\mathrm{diag}}

%% vectors
\newcommand{\vx}{\mathbf{x}}
\newcommand{\vy}{\mathbf{y}}
\newcommand{\vz}{\mathbf{z}}
\newcommand{\va}{\mathbf{a}}
\newcommand{\vb}{\mathbf{b}}
\newcommand{\vc}{\mathbf{c}}
\newcommand{\ve}{\mathbf{e}}
\newcommand{\vs}{\mathbf{s}}

%% matrices
\newcommand{\mA}{\mathbf{A}}
\newcommand{\mB}{\mathbf{B}}
\newcommand{\mC}{\mathbf{C}}
\newcommand{\mD}{\mathbf{D}}
\newcommand{\mE}{\mathbf{E}}
\newcommand{\mM}{\mathbf{M}}
\newcommand{\mP}{\mathbf{P}}

%% mutlivariate
\newcommand{\grad}{\mathrm{grad}}
\DeclareMathOperator*{\cart}{\scalebox{2}{\times}}

%% TikZ global styles
\tikzset{%
	thickaxis/.style={thick,->}
}

%% Vzhled beamer prezentace
\useoutertheme{infolines}
\useinnertheme[shadow]{rounded}
\usecolortheme{rose}
\usefonttheme{professionalfonts}

\setbeamertemplate{theorems}[ams style]

%% logo
\pgfdeclareimage[height=.7cm]{logo}{attachments/ctu_logo.pdf}
\logo{\pgfuseimage{logo}}

%% Titulni stranka
\title[HiRoPE]{HiRoPE: Length Extrapolation for Code Models Using Hierarchical Position}
\author[Maksim Sapronov]{%
    Authors: Kechi Zhang, Ge Li, Huangzhao Zhang, Zhi Jin \\
    \bigskip
    Maksim Sapronov
}
\institute[FIT CTU]{%
        Knowledge Engineering Seminar \\
        Faculty of Information Technology \\
        Czech Technical University in Prague
}
\date{}

\newtheoremstyle{presentation}%
	{0pt}% space above
	{0pt}% space below
	{\normalfont}% body font
	{}% indent amount
	{\bfseries}% theorem head font
	{:}% punctuation after theorem head
	{.5em}% space after theorem head
	{}% head spec (?)

\theoremstyle{presentation}
\newtheorem*{defi}{Definition}

\setbeamerfont*{block title example}{series=\bfseries,size=\normalsize,family=\sffamily,shape=\upshape}
\setbeamerfont*{block title}{series=\bfseries,size=\normalsize,family=\sffamily,shape=\upshape}

\makeatletter
\def\th@tweek{%
	\normalfont
	\def\inserttheoremblockenv{exampleblock}
}
\theoremstyle{tweek}
\newtheorem*{prik}{Příklad}
\makeatother

%% dalsi vylepseni
\setbeamertemplate{frametitle continuation}[from second][(pokračování)]
\setbeamertemplate{navigation symbols}{}
\usefonttheme[onlymath]{serif}
\setbeamertemplate{sections/subsections in toc}[square]

%% other
\setbeamertemplate{blocks}[rounded][shadow=true]
\setbeamertemplate{headline}{%
	\begin{beamercolorbox}{section in head/foot}
		\vskip2pt\insertnavigation{\paperwidth}\vskip2pt
	\end{beamercolorbox}
}
\setbeamertemplate{frametitle}{%
	\nointerlineskip
	\begin{beamercolorbox}[ht=1.6em,wd=\paperwidth,sep=0.2cm]{frametitle}
	\vbox{}\vskip-2ex%
	\strut\insertframetitle\strut
	\vskip-0.8ex%
	\end{beamercolorbox}
}

