\chapter*{Objectives of the Thesis}\addcontentsline{toc}{chapter}{Objectives of the Thesis}\markboth{Objectives of the Thesis}{Objectives of the Thesis}

The objectives of this thesis are divided into two parts, corresponding to the theoretical and practical aspects of the work.

The theoretical part aims to provide a comprehensive and high-level survey of the current state of the field related to project adaptation in the context of full line code completion, focusing on the capabilities and limitations of in-context learning and repository-level code completion methods. This part establishes the necessary background to understand the practical contributions of the thesis.

The practical part addresses several research questions concerning the impact of various context composition strategies on the quality of code completion, evaluated across three setups: a pre-trained Code Large Language Model (Code LLM), the same model fine-tuned with a specific context composition strategy, and the model after context window extension. To address these questions, the thesis describes a context composition framework for extracting relevant information from software repositories, as well as a fine-tuning pipeline for adapting Code LLMs to project-specific data. The outcomes characterize the role of context composition in enhancing code completion quality and provide insights into its integration within repository-level training procedures.
